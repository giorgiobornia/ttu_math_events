\documentclass[oneside]{article}
\usepackage{amssymb,amsfonts}
\usepackage{amsthm}

\usepackage{hyperref}

\usepackage{fancyhdr}

\usepackage[none]{hyphenat} 
\usepackage[left=.75 in,top=.75 in,right=.75 in,bottom=.75 in,nohead]{geometry}

\pagestyle{fancy}
\lfoot{}
\cfoot{\url{http://www.math.ttu.edu/~dacao/AnalysisSeminar}}
%\cfoot{\textit{http:/$\!$/www.math.ttu.edu/$\!$\raise-1ex\hbox{{\Large\texttt{\char`\~}}}lhoang/AppliedMath/}}
\rfoot{}
\fancyhead{}
\renewcommand{\headrulewidth}{0pt}
\linespread{1.5}

%%%%%%%%%%%%%%%%%%%%%%%%%%%%%%%%%%%%%%%%%%%%%%%%%%%%%%%%%%%%%%%%%%%%

\newcommand{\talktitle}{Regularity at infinity with respect to solution of  Mixed Boundary Value Problems for Elliptic Equations.}

\newcommand{\talkspeaker}{ \textbf{\sc Akif Ibragimov  }\\ \textit{Texas Tech University}}

\newcommand{\talkdate}{\textbf{Monday, April 16, 2018}}

\newcommand{\timelocation}{\textbf{Room: MATH 108.  Time: 4:00 pm.}}

\newcommand{\talkabstract}{
%
}

\begin{document}

%\vspace*{-2cm}
\begin{center}
{\LARGE
Texas Tech University.  Analysis Seminars.
}
\medskip

\textbf{\Huge {\uppercase{\talktitle}} }

{\LARGE
\talkspeaker\\
\talkdate\\
\timelocation
}
\end{center}

\vspace*{10pt}


%\addtolength{\linewidth}{-1cm}
\begin{center}

\fbox{\parbox{\linewidth}{
\begin{center}
\begin{minipage}[c]{.96\linewidth}
{
\vspace*{.25cm}
{\LARGE \textbf{ABSTRACT.}}
{\Large
We investigate the regularity at infinity of solution to the Zaremba type problem in unbounded  domain for the non-divergence elliptic equation. 
The main result sufficient test imposed on support of the Dirichlet Data, which guarantee asymptotic convergence of the solution to the value of the limiting value on the  boundary of Dirichlet condition. The result is formulated in terms of so-called $s$-capacity of the Dirichlet portion of the boundary, while the Neumann boundary should satisfy certain ``admissibility'' condition.
}
\vspace*{.25cm}
}
\end{minipage}
\end{center}
}}

\end{center}
\end{document}
