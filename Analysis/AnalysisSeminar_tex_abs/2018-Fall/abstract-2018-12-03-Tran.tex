\documentclass[oneside]{article}
\usepackage{amssymb,amsfonts}
\usepackage{amsthm}

\usepackage{hyperref}

\usepackage{fancyhdr}

\usepackage[none]{hyphenat} 
\usepackage[left=.75 in,top=.75 in,right=.75 in,bottom=.75 in,nohead]{geometry}

\pagestyle{fancy}
\lfoot{}
\cfoot{\url{http://www.math.ttu.edu/~dacao/AnalysisSeminar}}
%\cfoot{\textit{http:/$\!$/www.math.ttu.edu/$\!$\raise-1ex\hbox{{\Large\texttt{\char`\~}}}lhoang/AppliedMath/}}
\rfoot{}
\fancyhead{}
\renewcommand{\headrulewidth}{0pt}
\linespread{1.5}

%%%%%%%%%%%%%%%%%%%%%%%%%%%%%%%%%%%%%%%%%%%%%%%%%%%%%%%%%%%%%%%%%%%%

\newcommand{\talktitle}{Steklov Eigenvalues and Free Boundary Minimal Surfaces.}

\newcommand{\talkspeaker}{ \textbf{\sc Hung Tran }\\ \textit{Texas Tech University}}

\newcommand{\talkdate}{\textbf{Monday, December 03, 2018}}

\newcommand{\timelocation}{\textbf{Room: MATH 109.  Time: 4:00pm.}}

\newcommand{\talkabstract}{
%
}

\begin{document}

%\vspace*{-2cm}
\begin{center}
{\LARGE
Texas Tech University.  Analysis Seminars.
}
\medskip

\textbf{\Huge {\uppercase{\talktitle}} }

{\LARGE
\talkspeaker\\
\talkdate\\
\timelocation
}
\end{center}

\vspace*{10pt}


%\addtolength{\linewidth}{-1cm}
\begin{center}

\fbox{\parbox{\linewidth}{
\begin{center}
\begin{minipage}[c]{.96\linewidth}
{
\vspace*{.25cm}
{\LARGE \textbf{ABSTRACT.}}
{\Large 
The connection between Steklov eigenvalues and free boundary minimal surfaces has been developed to great effects in the last couple of years. In this talk, we first describe that connection between PDE and geometry. Then, we'll introduce the notion of a Jacobi-Steklov eigenvalue and see how it reveals extremal properties of a free boundary minimal surface. 
}
\vspace*{.25cm}
}
\end{minipage}
\end{center}
}}

\end{center}
\end{document}
