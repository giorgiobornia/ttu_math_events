\documentclass[oneside]{article}
\usepackage{amssymb,amsfonts}
\usepackage{amsthm}

\usepackage{hyperref}

\usepackage{fancyhdr}

\usepackage[none]{hyphenat} 
\usepackage[left=.75 in,top=.75 in,right=.75 in,bottom=.75 in,nohead]{geometry}

\pagestyle{fancy}
\lfoot{}
\cfoot{\url{http://www.math.ttu.edu/~dacao/AnalysisSeminar}}
%\cfoot{\textit{http:/$\!$/www.math.ttu.edu/$\!$\raise-1ex\hbox{{\Large\texttt{\char`\~}}}lhoang/AppliedMath/}}
\rfoot{}
\fancyhead{}
\renewcommand{\headrulewidth}{0pt}
\linespread{1.5}

%%%%%%%%%%%%%%%%%%%%%%%%%%%%%%%%%%%%%%%%%%%%%%%%%%%%%%%%%%%%%%%%%%%%

\newcommand{\talktitle}{Nonlinear operator splitting techniques for abstract stochastic problems. Part I.}

\newcommand{\talkspeaker}{ \textbf{\sc Joshua Padgett }\\ \textit{Texas Tech University}}

\newcommand{\talkdate}{\textbf{Monday, October 29, 2018}}

\newcommand{\timelocation}{\textbf{Room: MATH 109.  Time: 4:00pm.}}

\newcommand{\talkabstract}{
%
}

\begin{document}

%\vspace*{-2cm}
\begin{center}
{\LARGE
Texas Tech University.  Analysis Seminars.
}
\medskip

\textbf{\Huge {\uppercase{\talktitle}} }

{\LARGE
\talkspeaker\\
\talkdate\\
\timelocation
}
\end{center}

\vspace*{10pt}


%\addtolength{\linewidth}{-1cm}
\begin{center}

\fbox{\parbox{\linewidth}{
\begin{center}
\begin{minipage}[c]{.96\linewidth}
{
\vspace*{.25cm}
{\LARGE \textbf{ABSTRACT.}}
{\Large 
Operator splitting has long been a useful tool when approximating the solutions to differential equations. The consideration of nonlinear operator splitting was an important innovation for simplifying nonlinear problems, but one which was harder to justify until recent years. With the increasing need to consider stochastic problems, one hopes that these splitting techniques can be extended to this situation as well. In this talk, I will give an introduction to operator splitting and provide a justification to existing nonlinear splitting techniques via the consideration of the Magnus expansion. I will then demonstrate how this approach is preferable when considering coupled differential equations exhibiting both self- and cross-diffusion. After this, I will then introduce abstract stochastic problems and extend the notion of operator splitting to this setting. This approach will be clarified by considering novel results regarding the convergence of these splitting techniques in the appropriate operator norm. This talk should be accessible to most graduate students.
}
\vspace*{.25cm}
}
\end{minipage}
\end{center}
}}

\end{center}
\end{document}
