\documentclass[oneside]{article}
\usepackage{amssymb,amsfonts}
\usepackage{amsthm}

\usepackage{hyperref}

\usepackage{fancyhdr}

\usepackage[none]{hyphenat} 
\usepackage[left=.75 in,top=.75 in,right=.75 in,bottom=.75 in,nohead]{geometry}

\pagestyle{fancy}
\lfoot{}
\cfoot{\url{http://www.math.ttu.edu/~dacao/AnalysisSeminar}}
%\cfoot{\textit{http:/$\!$/www.math.ttu.edu/$\!$\raise-1ex\hbox{{\Large\texttt{\char`\~}}}lhoang/AppliedMath/}}
\rfoot{}
\fancyhead{}
\renewcommand{\headrulewidth}{0pt}
\linespread{1.5}

%%%%%%%%%%%%%%%%%%%%%%%%%%%%%%%%%%%%%%%%%%%%%%%%%%%%%%%%%%%%%%%%%%%%

\newcommand{\talktitle}{Asymptotic expansions for decaying solutions of ODEs. Part II.}

\newcommand{\talkspeaker}{ \textbf{\sc Luan Hoang }\\ \textit{Texas Tech University}}

\newcommand{\talkdate}{\textbf{Monday, April 15, 2019}}

\newcommand{\timelocation}{\textbf{Room: MATH 112.  Time: 4:00pm.}}

\newcommand{\talkabstract}{
%
}

\begin{document}

%\vspace*{-2cm}
\begin{center}
{\LARGE
Texas Tech University.  Analysis Seminars.
}
\medskip

\textbf{\Huge {\uppercase{\talktitle}} }

{\LARGE
\talkspeaker\\
\talkdate\\
\timelocation
}
\end{center}

\vspace*{10pt}


%\addtolength{\linewidth}{-1cm}
\begin{center}

\fbox{\parbox{\linewidth}{
\begin{center}
\begin{minipage}[c]{.96\linewidth}
{
\vspace*{.25cm}
{\LARGE \textbf{ABSTRACT.}}
{\Large 
We establish the asymptotic expansions, when time tends to infinity, for decaying solutions of non-linear non-autonomous systems of ordinary differential equations. This extends the original work of Foias-Saut for Navier-Stokes equations for potential forces (which deals with a bilinear mapping). In our study, the nonlinearity is more general, and can have an expansion form of any order.
In addition, the forcing term can decay, in time, exponentially or algebraically at arbitrary rates.
We prove that  any decaying solution admits an asymptotic expansion, as time goes to infinity, of the same type as the force's.
Particular, in case of the algebraic (power) decay, the expansion  is independent of the solution, as a contrast to the exponential case.
Examples of applications to solutions near special periodic orbits are also given. 
This  is joint work with Dat Cao (Texas Tech University).
}
\vspace*{.25cm}
}
\end{minipage}
\end{center}
}}

\end{center}
\end{document}
