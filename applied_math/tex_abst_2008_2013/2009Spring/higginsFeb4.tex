\documentclass[oneside]{amsart}
\usepackage{amssymb,amsfonts}
\usepackage{amsthm}

\linespread{2}

%%%%%%%%%%%%%%%%%%%%%%%%%%%%%%%%%%%%%%%%%%%%%%%%%%%%%%%%%%%%%%%%%%%%

\begin{document}
\begin{center}
Texas Tech University. Applied Mathematics Seminar.

\end{center}

\begin{center}

{\LARGE \uppercase{\textbf{
Generalized Exponential Function on a Time Scale
}}}

Raegan Higgins, TexasTech University

February 4, 2009

Room: MA 111, Time: 4:00pm

\end{center}

ABSTRACT.  A time scale T is just a closed nonempty subset of the real numbers. Time scales include the real numbers, the integers, and the Cantor set. Given a smooth function p(t) defined on a time scale and a point s in the time scale we will define a generalized exponential function ep(t,s) which generalizes the exponential function ept studied in calculus.
\end{document}
