\documentclass[oneside]{amsart}
\usepackage{amssymb,amsfonts}
\usepackage{amsthm}

\linespread{1.5}

%%%%%%%%%%%%%%%%%%%%%%%%%%%%%%%%%%%%%%%%%%%%%%%%%%%%%%%%%%%%%%%%%%%%

\begin{document}
\begin{center}
Texas Tech University.  Applied Mathematics Seminar.

COLLOQUIUM TALK
\end{center}

\begin{center}

{\LARGE \uppercase{\textbf{
Regularity of weak solutions to strongly coupled elliptic/parabolic systems via nonlinear heat approximation methods
}}}

Dung Le, University of Texas, San Antonio

Thursday, April 2, 2009

Room: MA 114, Time: 3:30pm

\end{center}

ABSTRACT. Nonlinear elliptic/parabolic partial differential systems have been used to model real life phenomena. Regularity theory of solutions to these systems plays a fundamental role in the mathematical and rigorous study of these models. A priori estimates for Holder norms of solutions will provide global existence, compactness property etc. for the flow of solutions in order to carry out further investigations on the dynamics of the solutions. Ad hoc techniques, such as DeGiorgi classes and Moser iteration methods and theirs generalizations, were well known tools to tackle regularity problems. These methods require deep and hard analysis tools, and can apply only to scalar equations. I will introduce an alternative approach using heat approximation techniques to derive a priori estimates for Holder norms of bounded or unbounded weak solutions. This method is quite elementary as it needs only basic knowledge in functional analysis, real analysis and partial differential equations. Most importantly, the theory can apply to large nonlinear systems in applications. The talk is accesible to graduate students. If time permits, I will also talk about the generalized versions of our method for fully nonlinear with degeneracy/singularity such as p-Laplacian or porous media type systems.
\end{document}
