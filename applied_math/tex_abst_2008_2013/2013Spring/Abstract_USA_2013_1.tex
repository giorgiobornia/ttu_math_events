\documentclass[12pt]{article}
\usepackage[cp1251]{inputenc} %��������� -[cp]866]- ��� DOS
\usepackage[english]{babel} %����������� ���.-����.
\usepackage{amssymb,amsfonts,amsmath,euscript}
%\usepackage{amsmath, amssymb,graphics,amsfonts, array, bm}%linedraw}
\pagestyle{myheadings}
\pagestyle{empty}
%\language=1
%\textwidth=108mm
%\textheight=165mm
%\topmargin=-2 true cm
\begin{document}

\begin{center}\bigskip
{\bf Igor~Bloshanskii}
\bigskip

{\it Lomonosov Moscow State University}\\
{\it Moscow State Regional University}\bigskip


\bigskip
{\bf Equiconvergence of expansions in multiple\\ trigonometric Fourier series and
Fourier integral in the case of "lacunary sequence of partial sums"\, }\,
\bigskip\bigskip

\end{center}
{\bf Abstract.}
The question under investigation is equiconvergence on $\mathbb T^N=[-\pi, \pi)^N$ of expansions in multiple trigonometric Fourier series and in Fourier integral of functions $f\in L_p({\mathbb T}^N)$ and $g\in L_p({\mathbb R}^N)$, $p>1$, $N\ge 2$,  $g(x)=f(x)$ on $\mathbb T^N$.  We consider the case when the rectangular "partial sums"\, of these expansions, i.e. $S_n(x;f)$ and $J_\alpha(x;g)$ correspondingly, have "indexes"\, $n=(n_1,\dots,n_N)\in {\mathbb Z}^N$  and $\alpha=(\alpha_1,\dots,\alpha_N)\in {\mathbb R}^N$ with components $n_j$ and $\alpha_j$ satisfying relation: $|\alpha_j-n_j|\leq C$, $j=1,\dots,N$,  constant $C$ does not depend on $n$ and $\alpha$. In particular, the case when some of components $n_j$ are elements of lacunary sequences is considered. An "almost"\, Cauchy property for sequences of rectangular partial sums of multiple Fourier expansions of functions in $L_p$, $p>1$ was found.







\enddocument





