\documentclass[oneside]{amsart}
\usepackage{amssymb,amsfonts}
\usepackage{amsthm}

\linespread{2}

%%%%%%%%%%%%%%%%%%%%%%%%%%%%%%%%%%%%%%%%%%%%%%%%%%%%%%%%%%%%%%%%%%%%

\begin{document}
\begin{center}
Texas Tech University.  Applied Mathematics Seminar.

\end{center}

\begin{center}

{\LARGE \uppercase{\textbf{
Localization Phenomena in Matrix Functions: Theory and Applications
}}}

Michele Benzi, Emory University

Friday, February 20, 2009 

Room: MA 114, Time: 10:00am

\end{center}

ABSTRACT. Many physical phenomena are characterized by strong localization, that is, 
rapid decay outside of a small spatial or temporal region. Frequently, 
such localization can be expressed as decay in the entries of a function 
f(A) of an appropriate sparse or banded matrix A that encodes a description 
of the system under study. Important examples include the decay in the 
entries of the density matrix for non-metallic systems in quantum chemistry 
(a function of the Hamiltonian), the localization of the eigenvectors in the 
Anderson model, and the decay behavior of the inverse of a banded symmetric 
positive definite matrix. Localization phenomena are of fundamental importance 
both at the level of the physical theory and at the computational level, 
because they open up the possibility of approximating relevant matrix 
functions in O(N) time, where N is a measure of the size of the system. 
In this talk I will give an overview of theoretical results, algorithms, 
and applications in various parts of computational physics and numerical 
analysis.

\end{document}
